Sulle due applicazioni sono stati eseguiti diversi \textit{test} e possiamo affermare che l'applicazione si comporta abbastanza bene. Certamente, il progetto potrebbe essere migliorato perchè i limiti individuati sono i seguenti:
\begin{itemize}
		\item Se si suonassero ad esempio, le note 64 e 65 della prima corda in sequenza, la rete potrebbe predire una delle due note o tutte e due non sulla stessa corda ma in quella successiva. Questo non sarebbe un errore perchè il suono è lo stesso ma in certi casi sarebbe meglio suonare tasti vicini per una questione di manualità. Questo perchè il sistema determina la tablatura finestra per finestra e non tiene conto della sequenza della registrazione;
		\item Se la canzone è troppo veloce, la predizione tende ad essere errata;
		\item Alcune note potrebbero non essere riconosciute perchè c'è ancora un margine di errore di quasi il 10\%.
	\end{itemize}
Nonostante ciò, siamo molto soddisfatti del progetto che è stato realizzato. Siamo consapevoli che abbiamo esplorato solo una piccola parte di questa vastissima materia ma quello che abbiamo appreso sarà sicuramente usato come base di partenza in futuri progetti.\\
\newline
A fini dimostrativi, forniamo insieme alla documentazione due brevi video dell'applicazione in funzione.