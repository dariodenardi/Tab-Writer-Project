Sulle due applicazioni sono stati eseguiti diversi \textit{test} e possiamo affermare che si comportano entrambi abbastanza bene. Certamente, il progetto potrebbe essere migliorato perchè i limiti individuati sono i seguenti:
\begin{itemize}
		\item Se si suonassero ad esempio, le note 64 e 65 della prima corda in sequenza, la rete potrebbe predire una delle due note, o tutte e due, non sulla stessa corda ma in quella successiva. Questo non sarebbe un errore perchè il suono è lo stesso ma in certi casi sarebbe meglio suonare tasti vicini per una questione di manualità. Questo perchè il sistema determina la tablatura finestra per finestra e non tiene conto della sequenza della registrazione;
		\item Se la canzone è troppo veloce, cioè ha un tempo o BPM (battiti per minuto) alto, la predizione tende ad essere errata;
		\item Alcune note potrebbero non essere riconosciute perchè c'è ancora un margine di errore di quasi il 10\%.
	\end{itemize}
Nonostante ciò, siamo molto soddisfatti del progetto che è stato realizzato. Grazie a questa materia abbiamo esplorato due mondi che per noi erano sconosciuti come lo sviluppo di applicazioni \textit{mobile}. Siamo consapevoli che abbiamo perlustrato solo una piccola parte di questa vastissima materia ma quello che abbiamo appreso sarà sicuramente usato come base di partenza per progetti futuri.\\
\newline
A fini dimostrativi, forniamo insieme alla documentazione due brevi video delle applicazioni in funzione.